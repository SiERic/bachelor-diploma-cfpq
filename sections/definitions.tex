\subsection{Ключевые понятия и термины}

\subsubsection{Понятия из теории графов}

% Помеченный граф
\begin{definition}
    \textit{Ориентированный помеченный граф} (или граф с метками)~--- это тройка $G = \langle V, E, \Sigma \rangle$, где $V$~--- множество вершин, $\Sigma$~--- множество меток, $E \subseteq V \times V \times \Sigma$~--- множество рёбер. 

    Неформально, это обычный мультиграф, каждому ребру которого сопоставлена метка из алфавита $\Sigma$.
\end{definition}

\begin{definition}[Транзитивное замыкание]

\TODO

\end{definition}

\begin{definition}[Инкрементальное транзитивное замыкание]

\TODO

\end{definition}

\subsubsection{Основные понятия из теории формальных языков}

% Контекстно-свободная грамматика
\begin{definition}
\textit{Контекстно-свободная грамматика}~--- это четвёрка $\langle \Sigma, N, S, P \rangle$, где
\begin{itemize}
    \item $\Sigma$~--- конечный алфавит
    \item $N$~--- конечное множество нетерминалов
    \item $S \in N$~--- стартовый нетерминал
    \item $P$~--- конечное множество продукций (правил грамматики), имеющих вид\\ $N_i \to \alpha$, где $N_i \in N, \alpha \in (\Sigma \cup N)^{*}$
\end{itemize}
\end{definition}

\begin{example}
\TODO: пример
\end{example}

% Контекстно-свободный язык
\begin{definition}
\textit{Контекстно-свободный язык}~--- это язык, распознаваемый контекстно-свободной грамматикой
\end{definition}

\begin{definition}[Конечный автомат (?)]
    НКА и ДКА
    
\end{definition}

\begin{definition}[Рекурсивный конечный автомат (РКА)]
    \textit{Для простоты тут будет немного не такое определение, как в~\cite{Alur05}}

    Это набор компонент $M_1, M_2, \dots , M_k$, где каждая компонента $M_i$~--- это пятёрка $\langle Q_i, \Sigma_i, En_i, Ex_1, \delta_i \rangle$, где 
        \begin{itemize}
            \item $Q_i$~--- конечное множество состояний
            \item $\Sigma_i$~--- конечный алфавит
            \item $En_i \subset Q_i$~--- множество начальных состояний
            \item $Ex_i \subset Q_i$~--- множество конечных состояний
            \item $\delta_i \colon Q_i \times (\Sigma_i \cup \bigcup\limits_{j = 1}^k En_i \times Ex_i ) \to Q_i$~--- функция перехода. У $\delta_i$ есть два типа переходов: \textit{внутренние}, которые работают как обычные переходы в НКА и \textit{рекурсивные}, которые делают вызов другой компоненты (при этом обозначая начальную и конечную вершину в ней).
        \end{itemize}

    Неформально, это набор компонент, каждая из которых представляет собой ДКА, на рёбрах которого могут быть ``рекурсивные вызовы'' других компонент.

    \TODO: картинка с примером
    
\end{definition}

\begin{definition}[Прямое произведение автоматов]

\TODO

\end{definition}

\subsubsection{Всякие специфичные для задачи штуки}

% Язык Дика
\begin{definition}
    \textit{Языком Дика} на $k$ типах скобок $(D_k)$ называют контекстно-свободный язык над алфавитом $\Sigma_k = \{ (_1, )_1, (_2, )_2 \dots (_k, )_k \}$, состоящий из правильных скобочных последовательностей на $k$ типах скобок.

    Задачу CFPQ для языка Дика называют также задачу Диковой достижимости (Dyck-reachability).
\end{definition}

%Смешанный язык Дика
\begin{definition}

\TODO

\end{definition}

\begin{definition}[Двунаправленный граф]
    Помеченный граф $G = \langle V, E, \Sigma_k \rangle$ называется двунаправленным (bidirected), если в нём для каждого ребра $\langle u, v, (_i \rangle$ найдётся противоположное ребро $\langle v, u, )_i \rangle$ и наоборот.

    Неформально, матрица смежности такого графа симметрична, и метки на симметричных рёбрах~--- это парные открывающая/закрывающая скобки.
\end{definition}

\subsubsection{Просто определения, я не придумала, как это что это как его называть}

\begin{definition}[Система Непересекающихся Множеств (СНМ)]
    \TODO
\end{definition}
