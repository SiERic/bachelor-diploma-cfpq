\section{Ключевые понятия и термины}

\TODO: нет такой главы, это всё по другим главам расползётся

\subsection{Основные понятия из теории формальных языков}



\begin{definition}[Конечный автомат (?)]
    НКА и ДКА
    
\end{definition}

\begin{definition}[Рекурсивный конечный автомат (РКА)]
    \textit{Для простоты тут будет немного не такое определение, как в~\cite{Alur05}}

    Это набор компонент $M_1, M_2, \dots , M_k$, где каждая компонента $M_i$~--- это пятёрка $\langle Q_i, \Sigma_i, En_i, Ex_1, \delta_i \rangle$, где 
        \begin{itemize}
          \setlength\itemsep{-0.1em}
          \item $Q_i$~--- конечное множество состояний
          \item $\Sigma_i$~--- конечный алфавит
          \item $En_i \subset Q_i$~--- множество начальных состояний
          \item $Ex_i \subset Q_i$~--- множество конечных состояний
          \item $\delta_i \colon Q_i \times (\Sigma_i \cup \bigcup\limits_{j = 1}^k En_i \times Ex_i ) \to Q_i$~--- функция перехода. У $\delta_i$ есть два типа переходов: \textit{внутренние}, которые работают как обычные переходы в НКА и \textit{рекурсивные}, которые делают вызов другой компоненты (при этом обозначая начальную и конечную вершину в ней).
        \end{itemize}

    Неформально, это набор компонент, каждая из которых представляет собой ДКА, на рёбрах которого могут быть ``рекурсивные вызовы'' других компонент.

    \TODO: картинка с примером
    
\end{definition}

\begin{definition}[Прямое произведение автоматов]

\TODO

\end{definition}

\subsection{Постановка задачи}

\subsubsection{Всякие специфичные для задачи штуки}

\begin{definition}[Транзитивное замыкание]

\TODO

\end{definition}

\begin{definition}[Инкрементальное транзитивное замыкание]

\TODO

\end{definition}

% Язык Дика
\begin{definition}
    \textit{Языком Дика} на $k$ типах скобок $(D_k)$ называют контекстно-свободный язык над алфавитом $\Sigma_k = \{ (_1, )_1, (_2, )_2 \dots (_k, )_k \}$, состоящий из правильных скобочных последовательностей на $k$ типах скобок.

    Задачу CFPQ для языка Дика называют также задачу Диковой достижимости (Dyck-reachability).
\end{definition}

%Смешанный язык Дика
\begin{definition}

\TODO

\end{definition}

\begin{definition}[Двунаправленный граф]
    Помеченный граф $G = \q{V, E, \Sigma_k}$ называется двунаправленным (bidirected), если в нём для каждого ребра $\q{u, v, (_i}$ найдётся противоположное ребро $\q{v, u, )_i}$ и наоборот.

    Неформально, матрица смежности такого графа симметрична, и метки на симметричных рёбрах~--- это парные открывающая/закрывающая скобки.
\end{definition}

\subsubsection{Просто определения, я не придумала, как это что это как его называть}

\begin{definition}[Система Непересекающихся Множеств (СНМ)]
    \TODO
\end{definition}
