% У введения нет номера главы
\section*{Введение}

% Задача поиска путей с контекстно-свободными ограничениями была сформулирована \cite{Yannakakis1990} Михалисом Янакакисом в 1990 году в терминах запросов к графовым базам данных. С тех 

\subsection*{Актуальность}

Графовые модели данных широко используются в различных областях, например, в биоинформатике~\cite{Sevon08}, анализе социальных сетей~\cite{Zarrinkalam14, Chaudhary16}, графовых базах данных~\cite{Medeiros18,Yannakakis1990} и разных видах статического анализа (\TODO: ссылки). 

Одной из важных задач в анализе графовых моделей данных является поиск путей с заданными ограничениями. Одним из способов задавать такие ограничения являются формальные языки: если на рёбрах графа написаны метки из фиксированного алфавита, то можно искать пути, конкатенация меток на которых принадлежит фиксированному языку~\cite{Barrett00}. Например, хорошо изучена задача поиска путей с ограничениями, заданными регулярными языками (\TODO: link). В этой же работе мы остановимся на классе контекстно-свободных языков (\TODO: link (?)), так как они позволяют решать более широкий класс задач.

Задача поиска путей с контекстно-свободными ограничениями, или, сокращённо, CFPQ\footnote{Context-Free Path Querying} была сформулирована Михалисом Яннакакисом в 1990 году~\cite{Yannakakis1990} в применении к запросам к декларативному языку Datalog~\cite{DatalogWiki, Ceri1989}. С тех пор было предложение множество алгоритмов для её решения, в основном, основанных на разных видах синтаксического анализа: алгоритм Репса~\cite{Reps97}, использующий метод, схожий с алгоритмом Кока-Янгера-Касами~\cite{Younger1967},
 алгоритм Хеллингса~\cite{Hellings15}, использующий аннотированные грамматики, и другие~\cite{Santos18,Azimov18, Medeiros18, Orachev20, Chaudhuri08}. 

 К сожалению, недавно Кёйперс и др. экспериментально показали~\cite{Kuijpers19}, что текущие методы не достаточно эффективны для использования на практике. Что не удивительно, так как все они имеют асимптотику $\O(n^3)$ (где $n$~--- размер входного графа, а размер грамматики~--- константа), и лучшее ускорение, которого можно добиться, уменьшает время работы лишь в $\O(\log n)$ раз~\cite{Chaudhuri06} (используя метод четырёх русских~\cite{Arlazarov70}). Более того, существует условная нижняя оценка~\cite{Heintze1997,Chatterjee17}, согласно которой не существует комбинаторного\footnote{Этот термин не вполне определен, но можно понимать его как ``не алгебраический''. В частности, комбинаторные алгоритмы не должны использовать деление и вычитание, так те пользуются особенностями алгебраических структур (а именно, существованием обратного)} субкубического\footnote{С временем работы $\O(n^{3 - \eps})$} алгоритма для задачи CFPQ.

 Всё вышесказанное приводит к тому, что имеет смысл разрабатывать алгоритмы для частных случаев задачи, имеющие лучшее время работы. 

 Данная работа нацелена \TODO

\subsection*{Постановка задачи и ключевые термины}

\TODO: \textit{термины уехали в следующую главу, этот раздел, наверное, можно упразднить}

% Для формальной постановки задачи потребуется ввести некоторые вспомогательные определения.

\begin{definition}
Слово $w$ \textit{читается} на пути $p$, если конкатенация меток вдоль $p$ образует $w$. 
\end{definition}

Теперь определим саму задачу.

\begin{definition}
% [Задача поиска путей с контекстно-свободными ограничениями]
    Входной граф: $G$

    Входная грамматика: $\cool{G}$

    РКА входной грамматики: $\cool{R}$

    \TODO

    % Написать про семантику: https://arxiv.org/pdf/1502.02242.pdf

\end{definition}

\subsection*{Цель и задачи}

\TODO

% \subsection*{Достигнутые результаты}

% Какие результаты, я же ничего не делала весь год..

\subsection*{Структура работы}

\TODO