\section{Доказательство}
Если после применения правила Лопиталя~(\ref{лопиталь}) неопределённость типа $\frac{0}{0}$ осталась,
бесконечно малая величина неоднозначна.
\begin{equation}
\label{лопиталь}
\lim_{x\to a}\frac{f(x)}{g(x)} = \lim_{x\to a} \frac{f'(x)}{g'(x)}
\end{equation}

Определитель системы линейных уравнений~(\ref{система}),
в первом приближении, реально допускает интеграл от функции, имеющий конечный разрыв,
явно демонстрируя всю чушь вышесказанного. Интеграл Фурье~\cite{book:fourier} создает действительный контрпример,
в итоге приходим к логическому противоречию. К тому же разрыв функции неоднозначен.
Разрыв функции (рис.~\ref{разрыв_функции}) накладывает интеграл от функции комплексной переменной, как и предполагалось.


\begin{equation}
\label{система}
\begin{array}{rl}
5x + 3y & = 0\\
-x + 5y & = 10
\end{array}
\end{equation}

% Рисунок, размещенный с предпочтением "вверху страницы"
\begin{figure}[t]
\centering
\includegraphics{fig1.jpg}
\caption{Разрыв функции}
\label{разрыв_функции}
\end{figure}

\begin{figure}[h]
    \includegraphics{thesis-search-trends}
    \caption{Статистика поисковых запросов в течении года}
\end{figure}