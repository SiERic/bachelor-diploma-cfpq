% \subsection{Алгоритм для двунаправленных графов и языка Дика}

% \subsubsection{Мотивация (?)}

 % Например, рассмотреть неориентированные графы.

Для неориентированных графов отношение достижимости симметрично и на самом деле это отношение ``принадлежать одной компоненте связности''. Поддерживать добавление рёбер и проверку связности в неориентированном графе может СНМ. 

\subsection{Алгоритм, основанный на неориентированном транзитивном замыкании}

\textit{Я буду называть его Алгоритм НП}

В листинге~\ref{algo:NP} приведён псевдокод Алгоритма НП.

% At first, we assume, that each RSM box (of given grammar) is acyclic (if not, we can replace cycles with recursive call (how? {\color{red}{TODO}}))
% Then we can topologically sort every box (so that for every arc $q_i \rightarrow v \Leftrightarrow u < v$).


% We use two data structures: queue $Q$, storing edges, that should be added, but have not been yet, and Disjoint Set $D$, maintaining components in Kronecker product graph.

% Firstly, we add all arcs from initial Kronecker product into $Q$ and then start to iterate over it. At each step we take next arc from queue and join corresponding sets in $D$. If they are in differents sets and adding the edge leads to appearance of new path $(q_s, u) \rightsquigarrow (q_f, v)$ from start to final terminal of some box $S$, we add new arc form $u$ to $v$ labeled with $S$. After that we add this arc into Kronecker product: we iterate over all arcs $q_i \rightarrow q_j$ in grammar $G$ labeled with $S$ and add new edge $(q_i, u) \rightarrow (q_j, v)$ into $Q$. 

\begin{algorithm}[H]
    \floatname{algorithm}{Listing}
    \begin{algorithmic}[1]
    \caption{Алгоритм достижимости для РКА, основанный на неориентированном ТЗ}
    \label{algo:NP}
    \Function{UndirectedRSMReachability}{$\cool{R}$}
    % \Function{RSMReachability2}{$\cool{R}$}
        \State{$A \gets$ Empty adjacency matrix}
        \State{$Q \gets$ Empty Queue}
        \State{$D \gets$ DSU($|\bigcup\limits_{i=1}^k Q_i|$)}
        \For{$i \in 1..k$}
            \For{$u \xrightarrow{c} v \in \delta_i$}
                \State{$Q.Push(\langle u, v, i \rangle)$}
            \EndFor
        \EndFor
        \While{$Q$ is not Empty}
            \State{$\langle u, v, i \rangle \gets Q.Pop()$}
            \If{$u \in En_i \wedge v \in En_i$}
                \Comment{Нашли новый путь}
                \State{$A \gets A \cup getEdges(i, u, v)$}
                \State{$Q.PushAll(getEdges(i, u, v))$}
                \State{$D.Join(u, v, Q)$}
                \Comment{Добавляем новые рёбра}
            \EndIf
        \EndWhile
    \State \Return $A$
    \EndFunction
    \end{algorithmic}
\end{algorithm}

Для реализации алгоритма используются две вспомогательные структуры данных: очередь $Q$, хранящая рёбра, которые были добавлены в граф, но ещё не обработаны (как и в оригинальном алгоритме П2), и СНМ $D$, поддерживающая компоненты связности и поиск новых путей $\langle$стартовое состояние$\to$конечное состояние$\rangle$. 

Опишем подробно структуру используемого СНМ (в листинге~\ref{algo:DSU} приведён псевдокод функций \textit{Join()} и \textit{Get()}).

\algblockdefx[Structure]{Structure}{EndStructure}
[1]{{\bf Structure} #1}
{}

\begin{algorithm}[H]
    \floatname{algorithm}{Listing}
    \begin{algorithmic}[1]
    \caption{Система Непересекающихся Множеств}
    \label{algo:DSU}
    \Structure{DisjointSetUnion}
        \Function{DisjointSetUnion}{$V$}
            \For{$v \in V$}
                \State{$P[v] \gets v$}
                \Comment{Предкок}
                \State{$R[v] \gets 0$}
                \Comment{Ранг}
                \State{$En[v] \gets \{v \}$}
                \Comment{Список стартовых вершин поддерева}
                \State{$Ex[v] \gets \{v \}$}
                \Comment{Список конечных вершин поддерева}
            \EndFor
        \EndFunction
    \EndStructure
    \end{algorithmic}
\end{algorithm}

\TODO: Дописать код

За основу взята стандартная реализация \cite{Hopcroft1973}, использующая подвешенные деревья. 

\TODO: \textit{Надо ли её расписывать?}

Дополнительно в корнях хранятся списки всех начальных и конечных состояний компоненты. При добавлении ребра в операции \textit{Join} перебираются все пары начальная/конечная вершина из двух компонент и соответствующие им рёбра добавляются в рабочую очередь $Q$. 

% To find new paths from start to final terminal quickly, we store for every set in $D$ all initial and final states in it. To get this information we use supporting methods $GetInitialStates()$ and $GetFinalStates()$. To get RSM box label by state id we use supporting method $GetLabelByState()$. To get all edges in grammar labeled with $S$ we use supporting method $GetEdgesByLabel()$. 

% {\color{red}{TODO}}: rewrite in other notations

% {\color{red}{TODO}}: write about dealing with eps-transitions.

\TODO: (подумать) можно ли добавлять сразу много рёбер и сжимать их дфсом (как Борувка)?

\subsection{Время работы}

\begin{theorem}
На РКА из $n$ состояний и $m^{*}$ рёбрах в транзитивном замыкании, Алгоритм НП отработает за время $\O(n + m* \alpha(m^{*} + n, n))$
\end{theorem}

\begin{proof}
~\\
\TODO: \textit{$m*$ джойнов, а проходы по спискам не долгие, так как каждый раз генерируем новое ребро}
\end{proof}

\subsection{Корректность для неориентированных графов и некоторых классов грамматик}



\subsection{Корректность для двунаправленных графов и языка Дика}

\begin{remark} \label{r1}
    For bidirected graphs the Dyck-reachability relation forms an equivalence, i.e., for all bidirected graphs $G$, for every pair of nodes $u$m $v$, we have that $v$ is Dyck-reachable from $u$ iff $u$ is Dyck-reachable from $v$.
\end{remark}

\begin{note}
    Dyck language RSM-grammar box (there is only one).
    
    \begin{figure}[H]
        \includegraphics[width=\linewidth]{img/dyck_box.png}
    \end{figure}
\end{note}


\begin{theorem}
The Algorithm \ref{alg:UndirectodTensor} works correctly on bidirected graphs and Dyck grammars.
\end{theorem}
\begin{proof}

We only need to prove, that for every pair $u, v \in V(\mathcal{G})$ there is a path $(q_s, u) \rightsquigarrow (q_{f_i}, v)$ in ({\bf directed}) Kronecker product $G \otimes \mathcal{G}$ 
iff 
there is such path $(q_s, u) \rightsquigarrow (q_{f_j}, v)$ in an {\bf undirected} product (there $q_s$ is the initial state and $q_{f_i}, q_{f_j}$ are some final states of $G$ respectively).

$\Rightarrow$ 

Obviously (if $(q_{f}, v)$ is reachable from $(q_{s}, u)$ by directed edges, it is all the more reachable by undirected edges).

$\Leftarrow$

At first, note that the Kronecker product $G \otimes \mathcal{G}$ forms some kind of a layered structure~--- $i$-th layer consists of vertices $(q_i, v)$, where $q_i$ is $i$-th RSM state. Because RSM is topologically sorted ({\color{red}{TODO}}), every edge $(q_i, u) \rightarrow (q_j, v)$ goes forward.

We will call path \textit{simple} if it visits every layer no more than once.

We prove the claim by induction on the $l$ (path length).

Clearly the result is true for $l \le 3$, because the only way to achive final vertex in $1, 2$ or $3$ edges is by a simple vertical path (which exists in the original graph too).

Otherwise (if $l \ge 4$), path is not simple. 

Consider the first flex point of the path, that is the vertex $(q_i, v)$ such that edges $(q_j, u) \rightarrow (q_i, v)$ and $(q_i, v) \rightarrow (q_k, w)$ are in the path and $j, k \le i$ (so, the path is convex at this point).

Looking at the grammar graph we can notice, that every state has indegree $\le 1$. So at the flex point there are actually two same-labeled edges (that is, $j = k$). 

There can be three different types of labels on those edges:

\begin{itemize}
    \item $\alpha_l$-label

        \textit{path: $(q_0, u) \rightarrow (q_i, v) \rightarrow (q_0, w) \rightarrow \dots \rightarrow (q_f, z)$.}

        Since $\alpha_l$-labeled edges could only be added on the initialization stage, 
        graph $\mathcal{G}$ contains edges $u \xrightarrow{\alpha_l} v$ and $w \xrightarrow{\alpha_l} v$. Notice, that cause $\mathcal{G}$ is bidirected, it also has to contain edges $v \xrightarrow{\overline{\alpha_l}} u$ and $v \xrightarrow{\overline{\alpha_l}} w$.

        Now we can notice, that $w$ is Dyck-reachable (by the path $\alpha_l \overline{\alpha_l}$) from $u$, so there is an $S$-labeled edge from $u$ to $w$. We can also conclude, that (by induction) there is a directed path from $(q_0, w)$ to $(q_f, z)$ (there $z$ is the end of the path and $q_f$ is some final state of $G$), so there is an $S$-labeled edge from $w$ to $z$. 

        Using this two observation we can construct a directed path from $u$ to $z$: $u \xrightarrow{S} w \xrightarrow{S} z$. 
    \item $S$-label

        \textit{path: $(q_0, a) \rightarrow \dots \rightarrow (q_j, u) \rightarrow (q_i, v) \rightarrow (q_j, w) \rightarrow \dots \rightarrow (q_f, z)$.}

        $\mathcal{G}$ contains $S$-labeled edges $u \xrightarrow{S} v$ and $w \xrightarrow{S} v$. Since $\mathcal{G}$ is bidirected, then by \ref{r1} $v \xrightarrow{S} u$ and $v \xrightarrow{S} w$. Combining $u \xrightarrow{S} v$ and $v \xrightarrow{S} w$ we get that $u \xrightarrow{S} w$. 

        No we want to sort of contract this edge, joining $u$ and $w$ (on the $j$-th level). Then we can get (by induction hypothesis) the directed path from $(q_0, a) \rightsquigarrow (q_f, z)$. If new path does not contain joined $uw$ vertex, then that's the answer. Otherwise we can split this vertex back, inserting between $u$ and $w$ the $S$-labeled path (that one, from $u \xrightarrow{S} w$ edge). We can do it, because the both of these paths form correctly matched parenthesis ({\color{red}{TODO}}: we can prove this using stack-based checking algorithm).

        \textit{Two other cases can be proved the same way, but I find it a little dishonest}

    \item $\overline{\alpha_l}$-label

        \textit{path: $(q_0, a) \rightarrow \dots \rightarrow (q_j, u) \rightarrow (q_f, v) \rightarrow (q_j, w) \rightarrow \dots \rightarrow (q_f, z)$.}

        Since $\alpha_l$-labeled edges could only be added on the initialization stage, 
        graph $\mathcal{G}$ contains edges $u \xrightarrow{\overline{\alpha_l}} v$ and $w \xrightarrow{\overline{\alpha_l}} v$. Notice, that cause $\mathcal{G}$ is bidirected, it also has to contain edges $v \xrightarrow{\alpha_l} u$ and $v \xrightarrow{\alpha_l} w$.

        By induction, we get that $a \xrightarrow{S} v$. Now we will construct a second part of the path: $(q_0, v) \xrightarrow{\alpha_l} (q_{j-1}, w) \xrightarrow{S} (q_j, w) \rightsquigarrow (q_f, z)$ ($q_{j-1}, w) \xrightarrow{S} (q_j, w)$ ~--- initial $S$-loop). By induction, we have directed simple version of this path, so $v \xrightarrow{S} z$. 

        Combining this two paths ($a \xrightarrow{S} v$ and $v \xrightarrow{S} z$) we get $a \xrightarrow{SS} z \Rightarrow a \xrightarrow{S} z$~--- desired path.
        
    \begin{figure}[H]
        \includegraphics[width=\linewidth]{img/th_proof_img.png}
    \end{figure}
\end{itemize}

\end{proof}