% У введения нет номера главы
\section*{Введение}

\subsection*{Актуальность}

Графовые модели данных широко используются в различных областях науки, например, в биоинформатике~\cite{Sevon08}, анализе социальных сетей~\cite{Zarrinkalam14, Chaudhary16}, графовых базах данных~\cite{Medeiros18,Yannakakis1990} и разных видах статического анализа~\cite{Reps1998}. 

Одной из важных задач в анализе графовых моделей данных является поиск путей с заданными ограничениями. Одним из способов задавать такие ограничения являются формальные языки: если на рёбрах графа написаны метки из фиксированного алфавита, то можно искать пути, конкатенация меток на которых принадлежит фиксированному языку~\cite{Barrett00}. Например, хорошо изучена задача поиска путей с ограничениями, заданными регулярными языками~\cite{Mendelzon1995}. В этой же работе мы остановимся на классе контекстно-свободных языков, так как они позволяют решать более широкий класс задач.

Задача поиска путей с контекстно-свободными ограничениями, или, сокращённо, CFPQ\footnote{Context-Free Path Querying} была впервые сформулирована в терминах запросов к графовым базам данных~\cite{Yannakakis1990}, но нашла применение и в прочих отраслях, использующих графовые модели. За более чем 30 лет, прошедших с тех пор, было предложено множество разных алгоритмов для её решения~\cite{Reps97, Hellings15, Santos18}, в большинстве своём основанных на различных методах синтаксического анализа. 

К сожалению, все существующие решения задачи в общем случае недостаточно эффективны для использования на практике~\cite{Kuijpers19}. Более того, существует условная нижная оценка~\cite{Heintze1997}, согласно которой, скорее всего, достаточно быстрых решений задачи CFPQ в общем случае и не существует.

Всё вышесказанное приводит к тому, что имеет смысл разрабатывать алгоритмы для частных случаев задачи, имеющие время работы лучше, чем общее решение. Для некоторых из этих случаев уже были построены подобные решения, например для языков Дика на двунаправленных графах~\cite{Yuan09,Chatterjee17}. Проблема существующих решений в том, что они слишком Ad hoc, то есть построены специально под конкретный частный случай, а потому применённые в них подходы и идеи невозможно переиспользовать при построении решений для других частных случаев.  

Данная работа нацелена на создание единого подхода к построению решений задачи для CFPQ и применение этого подхода для разработки на его основе новых решений для некоторых частных случаев задачи.

\subsection*{Постановка задачи}

% Помеченный граф
\begin{definition}
  \textit{Ориентированный помеченный граф} (или \textit{граф с метками})~--- это тройка $G = \q{V, E, \Sigma}$, где $V$~--- множество вершин, $\Sigma$~--- множество меток, \\$E \subseteq V \times V \times \Sigma$~--- множество рёбер. 

  Неформально, это мультиграф, каждому ребру которого сопоставлена метка из алфавита $\Sigma$.

\end{definition}
  \begin{figure}[H]
    \begin{minipage}[h]{0.5\linewidth}
      \begin{tikzpicture}[shorten >=1pt, auto]
        \node[state] (q_0)                      {$0$};
        \node[state] (q_1) [above right=of q_0] {$1$};
        \node[state] (q_2) [right=of q_0]       {$2$};
        \node[state] (q_3) [right=of q_2]       {$3$};
        \path[->]
          (q_0) edge node {$a$} (q_1)
          (q_1) edge node {$a$} (q_2)
          (q_2) edge node {$a$} (q_0)
          (q_2) edge[bend left, above]  node {$b$} (q_3)
          (q_3) edge[bend left, below]  node {$b$} (q_2);
      \end{tikzpicture}
      \caption{Помеченный граф с $\Sigma = \{ a, b \}$}
    \end{minipage}
    \hfill
    \begin{minipage}[h]{0.5\linewidth} 
        \begin{tikzpicture}[shorten >=1pt, auto]
          \node[state] (q_0)                      {$0$};
          \node[state] (q_1) [above right=of q_0] {$1$};
          \node[state] (q_2) [right=of q_0]       {$2$};
          \node[state] (q_3) [right=of q_2]       {$3$};
          \path[->]
            (q_0) edge[ultra thick] node {$a$} (q_1)
            (q_1) edge[ultra thick] node {$a$} (q_2)
            (q_2) edge node {$a$} (q_0)
            (q_2) edge[bend left, above, ultra thick]  node {$b$} (q_3)
            (q_3) edge[bend left, below, ultra thick]  node {$b$} (q_2);
        \end{tikzpicture}
        \caption{На пути $v_0 \to v_1 \to v_2 \to v_3 \to v_2$\\ читается слово $aabb$}
      \end{minipage}
      \label{tikz:labeled_graph}
  \end{figure}


\begin{definition}
Будем говорить, что слово $w$ \textit{читается} на пути $p$, если конкатенация меток вдоль $p$ образует $w$. 
\end{definition}

% Контекстно-свободная грамматика
\begin{definition}
\textit{Контекстно-свободная грамматика}~--- это четвёрка $\q{\Sigma, N, S, P}$:
\begin{itemize}
  \setlength\itemsep{-0.1em}
  \item $\Sigma$~--- конечный алфавит
  \item $N$~--- конечное множество нетерминалов
  \item $S \in N$~--- стартовый нетерминал
  \item $P$~--- конечное множество продукций (правил вывода), имеющих вид\\ $N_i \to \alpha$, где $N_i \in N, \alpha \in (\Sigma \cup N)^{*}$
\end{itemize}
\end{definition}

\TODO: пример

% Контекстно-свободный язык
\begin{definition}
\textit{Контекстно-свободный язык}~--- язык слов, распознаваемых контекстно-свободной грамматикой.

Для данной контекстно-свободной грамматики $\cool{G}$, соответствующий ей язык обозначается как $L(\cool{G})$. 
\end{definition}

Теперь определим саму задачу.

% Задача поиска путей с контекстно-свободными ограничениями
\begin{definition}
  Дан ориентированный помеченный граф $G$ и контекстно-свободная грамматика $\cool{G}$ над тем же алфавитом.

  \textit{Задача поиска путей с контекстно-свободными ограничениями (CFPQ)} (в реляционной семантике~\cite{Hellings16}) заключается в нахождении всех пар вершин $u, v \in V(G)$, таких что существует путь $p: u \path v$, на котором читается слово $w \in L(\cool{G})$.

\end{definition}

\TODO: пример


\subsection*{Цель и задачи}

Целью работы является получить решения для частных случаев в задаче поиска путей с контекстно-свободными ограничениями, основываясь на едином подходе.

Для её достижения решаются следующие задачи:
    \begin{itemize}
      \item Выбрать единый подход для решения задачи CFPQ
      \item Построить на его основе решения для следующих частных случаев:
        \begin{itemize}
          \item Язык Дика $\cool{D}_k$ и двунаправленные графы
          \item Язык Дика $\cool{D}_1$ (на одном типе скобок)
        \end{itemize}
      \item Применить выбранного подхода для решение задачи достижимости для смешанный языка Дика $\cool{D}_1 \odot \cool{D}_1$ на двунаправленных графах
    \end{itemize}

\TODO: Всё тут переписать

\subsection*{Структура работы}

В главе 1 определены ключевые термины и дана формальная постановка задачи.

В главе 2 приведена история развития области и анализ существующих решений для задачи CFPQ, а также обоснован выбор подхода к построению новых частичных решений.

В главе 3 представлен выбранный подход к решению задачи CFPQ (идея пересечения языков).

\TODO