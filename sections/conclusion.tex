\section*{Заключение}

% \includegraphics[width=0.75\linewidth]{img/conclusion_goal} :(

Главным результатом данной работы является новый подход для разработки частных решений для задачи поиска путей с контекстно-свободными ограничениями. Подход заключается в сведении задачи поиска путей с контекстно-свободными ограничениями к решению задачи достижимости на рекурсивном автомате, распознающем пересечение языков входной грамматики и входного графа. В общем случае задача достижимости для РКА решается поддержанием инкрементального транзитивного замыкания и имеет ту же асимптотику, что и все остальные решения задачи CFPQ (а именно, $\O(|V|^3)$), однако оно может быть модифицировано для получения более быстрых решений в частных случаях.

Первой такой модификацией стало решение~\ref{algo:NP}, основанное на поддержании неориентированного инкрементального транзитивного замыкания (вместо ориентированного). Такое решение будет, в частности, работать, если РКА пересечения является неориентированным, но не только. Более практичным применением такого алгоритма является решение задачи Диковой достижимости на двунаправленных графах (теорема~\ref{thm:bidirected}). 

Ещё одной модификацией является алгоритм~\ref{algo:P}, основанный на пересчёте транзитивного замыкания каждый раз с нуля, а не итеративном обновлением при добавлении очередного ребра. Такой алгоритм будет иметь субкубическое время работы, если число итераций пересчёта транзитивного замыкания является небольшим. Например, такое алгоритм имеет время работы $\O(|V|^{\omega} \log^3 |V|)$ для языка Дика на одном типе скобок (теорема~\ref{thm:dyck_1}).

Идея пересечения языком может быть применена не только для задачи поиска путей с контекстно-свободными ограничениями. Так, в главе~\ref{section:dyck_1_1} этот подход применяется для решения задачи $\cool{D}_1 \odot \cool{D}_1$-достижимости, то есть для языка, который не является контекстно-свободным.

В качестве продолжения этой работы возможно рассмотрение применения данного подхода для других частных случаев задачи поиска путей с контекстно-свободными ограничениями, например, для планарных графов.