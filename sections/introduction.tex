% У введения нет номера главы
\section*{Введение}

% Задача поиска путей с контекстно-свободными ограничениями была сформулирована \cite{Yannakakis1990} Михалисом Янакакисом в 1990 году в терминах запросов к графовым базам данных. С тех 

\subsection*{Актуальность}

\subsection*{Постановка задачи и ключевые термины}

Для формальной постановки задачи потребуется ввести некоторые вспомогательные определения.

\begin{definition}[Помеченный граф]
    Ориентированный помеченный граф (edge-labeled graph) (или граф с метками)~--- это тройка $G = \langle V, E, \Sigma \rangle$, где $V$~--- множество вершин, $\Sigma$~--- множество меток, $E \subseteq V \times V \times \Sigma$~--- множество рёбер. 

    Неформально, это обычный мультиграф, каждому ребру которого сопоставлена метка из алфавита $\Sigma$.
\end{definition}

\begin{definition}[Контекстно-свободная грамматика]

\end{definition}

\begin{definition}[Контекстно-свободный язык]
    Язык, распознаваемый контекстно-свободной грамматикой

\end{definition}

Теперь определим саму задачу.

\begin{definition}[Задача поиска путей с контекстно-свободными ограничениями]
    Входной граф: $G$

    Входная грамматика: $\cool{G}$

    РКА входной грамматики: $\cool{R}$

    % Написать про семантику: https://arxiv.org/pdf/1502.02242.pdf

\end{definition}

Теперь будут введены некоторые понятие (в основном, из теории формальных языков), которые встретятся далее по тексту работы.

\begin{definition}[Язык Дика]
    Языком Дика на $k$ типах скобок $(D_k)$ называют контекстно-свободный язык над алфавитом $\Sigma_k = \{ \alpha_1, \overline{\alpha}_1, \dots \alpha_k, \overline{\alpha}_k \}$, состоящий из правильных скобочных последовательностей на $k$ типах скобок ($\alpha_i$ соответствует открывающей скобке, $\overline{\alpha}_i$~--- закрывающей).

    Задачу CFPQ для языка Дика называют также задачу Диковой достижимости (Dyck-reachability).
\end{definition}

\begin{definition}[Двунаправленный граф]
    Помеченный граф $G = \langle V, E, \Sigma_k \rangle$ называется двунаправленным (bidirected), если в нём для каждого ребра $(u, v, \alpha_i)$ найдётся противоположное ребро $(v, u, \overline{\alpha}_i)$ и наоборот.

    Неформально, матрица смежности такого графа симметрична, и метки на симметричных рёбрах~--- это парные открывающая/закрывающая скобки.
\end{definition}

\begin{definition}[Система Непересекающихся Множеств (СНМ)]

\end{definition}

\subsection*{Цель и задачи}

\subsection*{Достигнутые результаты}

\subsection*{Структура работы}